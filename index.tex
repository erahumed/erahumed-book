% Options for packages loaded elsewhere
\PassOptionsToPackage{unicode}{hyperref}
\PassOptionsToPackage{hyphens}{url}
\PassOptionsToPackage{dvipsnames,svgnames,x11names}{xcolor}
%
\documentclass[
  letterpaper,
  DIV=11,
  numbers=noendperiod]{scrreprt}

\usepackage{amsmath,amssymb}
\usepackage{iftex}
\ifPDFTeX
  \usepackage[T1]{fontenc}
  \usepackage[utf8]{inputenc}
  \usepackage{textcomp} % provide euro and other symbols
\else % if luatex or xetex
  \usepackage{unicode-math}
  \defaultfontfeatures{Scale=MatchLowercase}
  \defaultfontfeatures[\rmfamily]{Ligatures=TeX,Scale=1}
\fi
\usepackage{lmodern}
\ifPDFTeX\else  
    % xetex/luatex font selection
\fi
% Use upquote if available, for straight quotes in verbatim environments
\IfFileExists{upquote.sty}{\usepackage{upquote}}{}
\IfFileExists{microtype.sty}{% use microtype if available
  \usepackage[]{microtype}
  \UseMicrotypeSet[protrusion]{basicmath} % disable protrusion for tt fonts
}{}
\makeatletter
\@ifundefined{KOMAClassName}{% if non-KOMA class
  \IfFileExists{parskip.sty}{%
    \usepackage{parskip}
  }{% else
    \setlength{\parindent}{0pt}
    \setlength{\parskip}{6pt plus 2pt minus 1pt}}
}{% if KOMA class
  \KOMAoptions{parskip=half}}
\makeatother
\usepackage{xcolor}
\setlength{\emergencystretch}{3em} % prevent overfull lines
\setcounter{secnumdepth}{5}
% Make \paragraph and \subparagraph free-standing
\makeatletter
\ifx\paragraph\undefined\else
  \let\oldparagraph\paragraph
  \renewcommand{\paragraph}{
    \@ifstar
      \xxxParagraphStar
      \xxxParagraphNoStar
  }
  \newcommand{\xxxParagraphStar}[1]{\oldparagraph*{#1}\mbox{}}
  \newcommand{\xxxParagraphNoStar}[1]{\oldparagraph{#1}\mbox{}}
\fi
\ifx\subparagraph\undefined\else
  \let\oldsubparagraph\subparagraph
  \renewcommand{\subparagraph}{
    \@ifstar
      \xxxSubParagraphStar
      \xxxSubParagraphNoStar
  }
  \newcommand{\xxxSubParagraphStar}[1]{\oldsubparagraph*{#1}\mbox{}}
  \newcommand{\xxxSubParagraphNoStar}[1]{\oldsubparagraph{#1}\mbox{}}
\fi
\makeatother


\providecommand{\tightlist}{%
  \setlength{\itemsep}{0pt}\setlength{\parskip}{0pt}}\usepackage{longtable,booktabs,array}
\usepackage{calc} % for calculating minipage widths
% Correct order of tables after \paragraph or \subparagraph
\usepackage{etoolbox}
\makeatletter
\patchcmd\longtable{\par}{\if@noskipsec\mbox{}\fi\par}{}{}
\makeatother
% Allow footnotes in longtable head/foot
\IfFileExists{footnotehyper.sty}{\usepackage{footnotehyper}}{\usepackage{footnote}}
\makesavenoteenv{longtable}
\usepackage{graphicx}
\makeatletter
\def\maxwidth{\ifdim\Gin@nat@width>\linewidth\linewidth\else\Gin@nat@width\fi}
\def\maxheight{\ifdim\Gin@nat@height>\textheight\textheight\else\Gin@nat@height\fi}
\makeatother
% Scale images if necessary, so that they will not overflow the page
% margins by default, and it is still possible to overwrite the defaults
% using explicit options in \includegraphics[width, height, ...]{}
\setkeys{Gin}{width=\maxwidth,height=\maxheight,keepaspectratio}
% Set default figure placement to htbp
\makeatletter
\def\fps@figure{htbp}
\makeatother
% definitions for citeproc citations
\NewDocumentCommand\citeproctext{}{}
\NewDocumentCommand\citeproc{mm}{%
  \begingroup\def\citeproctext{#2}\cite{#1}\endgroup}
\makeatletter
 % allow citations to break across lines
 \let\@cite@ofmt\@firstofone
 % avoid brackets around text for \cite:
 \def\@biblabel#1{}
 \def\@cite#1#2{{#1\if@tempswa , #2\fi}}
\makeatother
\newlength{\cslhangindent}
\setlength{\cslhangindent}{1.5em}
\newlength{\csllabelwidth}
\setlength{\csllabelwidth}{3em}
\newenvironment{CSLReferences}[2] % #1 hanging-indent, #2 entry-spacing
 {\begin{list}{}{%
  \setlength{\itemindent}{0pt}
  \setlength{\leftmargin}{0pt}
  \setlength{\parsep}{0pt}
  % turn on hanging indent if param 1 is 1
  \ifodd #1
   \setlength{\leftmargin}{\cslhangindent}
   \setlength{\itemindent}{-1\cslhangindent}
  \fi
  % set entry spacing
  \setlength{\itemsep}{#2\baselineskip}}}
 {\end{list}}
\usepackage{calc}
\newcommand{\CSLBlock}[1]{\hfill\break\parbox[t]{\linewidth}{\strut\ignorespaces#1\strut}}
\newcommand{\CSLLeftMargin}[1]{\parbox[t]{\csllabelwidth}{\strut#1\strut}}
\newcommand{\CSLRightInline}[1]{\parbox[t]{\linewidth - \csllabelwidth}{\strut#1\strut}}
\newcommand{\CSLIndent}[1]{\hspace{\cslhangindent}#1}

\KOMAoption{captions}{tableheading}
\makeatletter
\@ifpackageloaded{bookmark}{}{\usepackage{bookmark}}
\makeatother
\makeatletter
\@ifpackageloaded{caption}{}{\usepackage{caption}}
\AtBeginDocument{%
\ifdefined\contentsname
  \renewcommand*\contentsname{Table of contents}
\else
  \newcommand\contentsname{Table of contents}
\fi
\ifdefined\listfigurename
  \renewcommand*\listfigurename{List of Figures}
\else
  \newcommand\listfigurename{List of Figures}
\fi
\ifdefined\listtablename
  \renewcommand*\listtablename{List of Tables}
\else
  \newcommand\listtablename{List of Tables}
\fi
\ifdefined\figurename
  \renewcommand*\figurename{Figure}
\else
  \newcommand\figurename{Figure}
\fi
\ifdefined\tablename
  \renewcommand*\tablename{Table}
\else
  \newcommand\tablename{Table}
\fi
}
\@ifpackageloaded{float}{}{\usepackage{float}}
\floatstyle{ruled}
\@ifundefined{c@chapter}{\newfloat{codelisting}{h}{lop}}{\newfloat{codelisting}{h}{lop}[chapter]}
\floatname{codelisting}{Listing}
\newcommand*\listoflistings{\listof{codelisting}{List of Listings}}
\makeatother
\makeatletter
\makeatother
\makeatletter
\@ifpackageloaded{caption}{}{\usepackage{caption}}
\@ifpackageloaded{subcaption}{}{\usepackage{subcaption}}
\makeatother

\ifLuaTeX
  \usepackage{selnolig}  % disable illegal ligatures
\fi
\usepackage{bookmark}

\IfFileExists{xurl.sty}{\usepackage{xurl}}{} % add URL line breaks if available
\urlstyle{same} % disable monospaced font for URLs
\hypersetup{
  pdftitle={ERAHUMED DSS},
  pdfauthor={Valerio Gherardi; Pablo Amador Crespo; Andreu Rico},
  colorlinks=true,
  linkcolor={blue},
  filecolor={Maroon},
  citecolor={Blue},
  urlcolor={Blue},
  pdfcreator={LaTeX via pandoc}}


\title{ERAHUMED DSS}
\author{Valerio Gherardi \and Pablo Amador Crespo \and Andreu Rico}
\date{}

\begin{document}
\maketitle

\renewcommand*\contentsname{Table of contents}
{
\hypersetup{linkcolor=}
\setcounter{tocdepth}{2}
\tableofcontents
}

\bookmarksetup{startatroot}

\chapter*{Preface}\label{preface}
\addcontentsline{toc}{chapter}{Preface}

\markboth{Preface}{Preface}

The purpose of this book is to provide a comprehensive reference for the
\href{https://www.erahumed.com/decision-support-system/}{ERAHUMED
Decision Support System}. Here you can find the technical descriptions
of the algorithms employed by the system, as well as the user manual for
the accompanying software.

The Support System and, hence, this book are currently under development
on \href{https://github.com/erahumed}{Github}. In particular, the
\texttt{\{erahumed\}} R package is hosted
\href{https://github.com/erahumed/erahumed}{here}.

For general information on the ERAHUMED project, please refer to the
\href{https://www.erahumed.com/}{official website}. If you want to get
in touch, you can contact any of us via e-mail:

\begin{itemize}
\tightlist
\item
  \href{mailto:andreu.rico@uv.es}{Andreu Rico (Coordinator)}
\item
  \href{mailto:pamadorc@gmail.com}{Pablo Amador (PhD Researcher)}
\item
  \href{mailto:vgherard840@gmail.com}{Valerio Gherardi (Software
  Developer)}
\end{itemize}

\bookmarksetup{startatroot}

\chapter{Introduction}\label{introduction}

This is a book created from markdown and executable code.

See Martı́nez-Megı́as et al. (2024) for additional info.

\part{Technical description}

\chapter{The ERAHUMED model: a bird's eye
view}\label{sec-birds-eye-view}

\chapter{ERAHUMED model components}\label{erahumed-model-components}

This Chapter provides detailed descriptions of the various components of
the ERAHUMED model, briefly introduced in
Chapter~\ref{sec-birds-eye-view}.

\section{INP: Input Data}\label{sec-inp}

The purpose of this model component is simply to collect the empirical
data that provides the observational input to all subsequent modeling
layers. This data consists of hydrological and meteorological time
series data for the Albufera lake in the desired time frame.

\subsection{Input}\label{input}

The only inputs to this modeling layer are the two aforementioned
time-series data-sets.

Hydrological data consists of the direct measurements of the Albufera
lake's daily water levels and outflows. Since the lake has three
estuaries (\emph{Gola de Pujol}, \emph{Gola del Perellonet} and
\emph{Gola del Perelló}), this data amounts to four daily time series.
The actual dataset bundled with the ERAHUMED software was obtained from
the public data repository compiled by the \emph{Confederación
Hidrográfica del Júcar} (TODO: Ref.), and is described in more detail in
Appendix~\ref{sec-data-hydro}.

Meteorological data consists of the daily measurements of precipitation
and evapotranspiration per unit are, and temperature (average, maximum
and minimum). This amounts to five daily the series. The actual data
bundled with the ERAHUMED software was obtained (TODO: where), and is
described in more detail in Appendix~\ref{sec-data-meteo}.

The time-series inputs described above are collected in the table below
(the common index \(t\) refers to the time - \emph{i.e.} day - of
observation).

\begin{longtable}[]{@{}
  >{\raggedright\arraybackslash}p{(\columnwidth - 4\tabcolsep) * \real{0.3553}}
  >{\raggedright\arraybackslash}p{(\columnwidth - 4\tabcolsep) * \real{0.1842}}
  >{\raggedright\arraybackslash}p{(\columnwidth - 4\tabcolsep) * \real{0.4605}}@{}}
\caption{Observational input to the ERAHUMED model (INP model
component).}\label{tbl-inp-time-series}\tabularnewline
\toprule\noalign{}
\begin{minipage}[b]{\linewidth}\raggedright
Input
\end{minipage} & \begin{minipage}[b]{\linewidth}\raggedright
Units
\end{minipage} & \begin{minipage}[b]{\linewidth}\raggedright
Description
\end{minipage} \\
\midrule\noalign{}
\endfirsthead
\toprule\noalign{}
\begin{minipage}[b]{\linewidth}\raggedright
Input
\end{minipage} & \begin{minipage}[b]{\linewidth}\raggedright
Units
\end{minipage} & \begin{minipage}[b]{\linewidth}\raggedright
Description
\end{minipage} \\
\midrule\noalign{}
\endhead
\bottomrule\noalign{}
\endlastfoot
\(L_t\) & \(\text{m}\) & Lake water level \\
\(O_t^{\text{Pujol}}\) & \(\text{m}^3\) & \emph{Pujol} daily outflow \\
\(O_t^{\text{Perelló}}\) & \(\text{m}^3\) & \emph{Perelló} daily
outflow \\
\(O_t^{\text{Perellonet}}\) & \(\text{m}^3\) & \emph{Perellonet} daily
outflow \\
\(\text{P}_t\) & \(\text{mm}\) & Precipitation (per unit area) \\
\(\text{ETP}_t\) & \(\text{mm}\) & Evapotranspiration (per unit area) \\
\(T_t^{\text{max}}\) & \(\text{°C}\) & Maximum temperature \\
\(T_t^{\text{min}}\) & \(\text{°C}\) & Minimum temperature \\
\(T_t^{\text{ave}}\) & \(\text{°C}\) & Average temperature \\
\end{longtable}

\subsection{Output}\label{output}

This modeling layer does not involve any actual computation, and its
output is simply a time-series data-set obtained as the combination of
the two input data-sets, \emph{i.e.} the collection of time series of
Table~\ref{tbl-inp-time-series}.

\section{HBA: Hydrological Balance of the Albufera lake}\label{sec-hba}

The purpose of this model component is to compute the total daily inflow
to the Albufera lake. The relevant equation expressing the hydrological
balance is:

\begin{equation}\phantomsection\label{eq-hydro-balance-symbolic}{
\text{Volume Change} = \text{Inflow} - \text{Outflow} + \text{Precipitation} - \text{Evapotranspiration}
}\end{equation}

where the unknown is \(\text{Inflow}\), whereas the remaining terms are
obtained from observational data, as described in the previous Section
(Section~\ref{sec-inp}).

\subsection{Input}\label{input-1}

The inputs specific to this layer are two numerical functions converting
water heights into the water \emph{volumes} that appear in
Equation~\ref{eq-hydro-balance-symbolic}. Specifically:

\begin{itemize}
\tightlist
\item
  The lake's \emph{storage curve} converts the measured lake's level
  into a total water volume.
\item
  The \emph{precipitation-evapotranspiration volume function} (or, for
  the sake of brevity, \emph{P-ETP function}), that converts
  precipitation and evapotranspiration values \emph{per unit area} into
  an overall water volume difference.
\end{itemize}

Even though the R interface to the ERAHUMED software allows arbitrary
definitions, the graphical user interface assumes a linear approximation
for both these functions. Therefore, the storage curve is assumed to
take the form:

\begin{equation}\phantomsection\label{eq-storage-function}{
V_t = m \cdot L_t + q,
}\end{equation} where \(V_t\) is the (daily) water volume of the lake,
\(L_t\) the corresponding water level (\emph{cf.}
Table~\ref{tbl-inp-time-series}), and \(m\) and \(q\) are numerical
coefficients\footnote{In fact, for the purpose of computing volume
  \emph{changes} entering the hydrological balance
  Equation~\ref{eq-hydro-balance-symbolic}, only the slope \(m\) is
  required.}. Correspondingly, the P-ETP function reads:

\begin{equation}\phantomsection\label{eq-petp-function}{
\Delta V_t ^{\text{P-ETP}} = \alpha \cdot \text{P}_t - \beta\cdot \text{ETP}_t,
}\end{equation} where the left-hand side represents the relevant water
volume difference, whereas the \(\text{P}\) and \(\text{ETP}\) terms in
the right-hand side are the measured precipitation and
evapotranspiration levels per unit area (\emph{cf.}
Table~\ref{tbl-inp-time-series}). The actual default values of \(m\),
\(q\), \(\alpha\) and \(\beta\) are documented in
Appendix~\ref{sec-storage-curve}.

\subsection{Output}\label{output-1}

The output of this model consists of the time-series collected in the
table below.

\begin{longtable}[]{@{}
  >{\raggedright\arraybackslash}p{(\columnwidth - 4\tabcolsep) * \real{0.3553}}
  >{\raggedright\arraybackslash}p{(\columnwidth - 4\tabcolsep) * \real{0.1842}}
  >{\raggedright\arraybackslash}p{(\columnwidth - 4\tabcolsep) * \real{0.4605}}@{}}
\caption{HBA model component
outputs}\label{tbl-hba-output}\tabularnewline
\toprule\noalign{}
\begin{minipage}[b]{\linewidth}\raggedright
Variable
\end{minipage} & \begin{minipage}[b]{\linewidth}\raggedright
Units
\end{minipage} & \begin{minipage}[b]{\linewidth}\raggedright
Description
\end{minipage} \\
\midrule\noalign{}
\endfirsthead
\toprule\noalign{}
\begin{minipage}[b]{\linewidth}\raggedright
Variable
\end{minipage} & \begin{minipage}[b]{\linewidth}\raggedright
Units
\end{minipage} & \begin{minipage}[b]{\linewidth}\raggedright
Description
\end{minipage} \\
\midrule\noalign{}
\endhead
\bottomrule\noalign{}
\endlastfoot
\(V_t\) & \(\text{m}^3\) & Lake water volume \\
\(I_t\) & \(\text{m}^3\) & Lake total inflow \\
\(O_t\) & \(\text{m}^3\) & Lake total outflow \\
\end{longtable}

\subsection{Details}\label{details}

Using the notation introduced in the previous Subsections, we can
rewrite Equation~\ref{eq-hydro-balance-symbolic} as follows:

\begin{equation}\phantomsection\label{eq-hydro-balance-explicit}{
V_{t+1} - V_t = I_t -O_t +\Delta V^{\text {ETP}}_t,
}\end{equation}

where \(V_t\) and \(\Delta V^{\text {ETP}}_t\) are computed through the
storage and P-ETP functions, as in Eqs. \ref{eq-storage-function} and
\ref{eq-petp-function}. The only terms that requires further
clarification in Equation~\ref{eq-hydro-balance-explicit} is \(O_t\),
the total outflow from the lake.

Strictly speaking, \(O_t\) is not merely the sum of
\(O_t^{\text{Pujol}}\), \(O_t^{\text{Perelló}}\) and
\(O_t^{\text{Perellonet}}\), but is rather calculated as follows:

\begin{equation}\phantomsection\label{eq-total-outflow}{
O_t = \max
\left[
O_t^{\text{Pujol}} + O_t^{\text{Perelló}} + O_t^{\text{Perellonet}},\, 
\Delta V^{\text {ETP}}_t+V_t-V_{t+1}
\right].
}\end{equation}

The rationale is that the simple sum of estuaries outflows omits
potentially important contributions from \emph{water recirculation},
that is to say, water being pumped out from the lake for rice-field
irrigation, by the so-called \emph{tancats}. Such amount of recirculated
water is hard to estimate and, in the lack of a better model, we simply
assume this to be negligible, \emph{except} when a positive amount is
required by Equation~\ref{eq-hydro-balance-explicit} itself, due the
physical constraint that \(I_t \geq 0\).

Once \(O_t\) is calculated through Equation~\ref{eq-total-outflow},
\(I_t\) can be immediately obtained from
Equation~\ref{eq-hydro-balance-explicit}. Notice that whenever the
aforementioned compensating outflow term due to water recirculation is
included (which happens when the maximum in Eq. \ref{eq-total-outflow}
is given by the second term), the total inflow is always estimated to be
zero.

\section{HBP: Hydrological Balance of rice Paddy
clusters}\label{hbp-hydrological-balance-of-rice-paddy-clusters}

This modeling layer simulates the local hydrology (\emph{i.e.} water
levels, inflows and outflows) of rice paddy clusters in the Albufera
National Park. The simulation, that literally involves the generation of
a synthetic data-set, is based on:

\begin{itemize}
\tightlist
\item
  The hydrology of the Albufera lake, which allows to constrain the
  total outflow of all rice paddies, assumed to be equal to the lake's
  total inflow (the main output of the HBA layer, see
  Section~\ref{sec-hba}).
\item
  An estimate of the fraction of the total lake's inflow that comes from
  each of the 26 ditches that flow into the Albufera.
\item
  A subdivision of the rice fields surface into clusters that are
  assumed to be hydrologically connected with a single ditch - meaning
  that the sum of outflows of clusters from the same group must equal
  the total flow through the corresponding ditch.
\item
  An ideal yearly management plan for the irrigation and draining of
  rice paddies.
\end{itemize}

\subsection{Input}\label{input-2}

As already mentioned, the simulation is based on a spatial clustering,
which is described in greater detail in
Appendix~\ref{sec-data-rice-clusters}. For the present purposes, we
stress that each cluster, consists of rice paddies of a definite rice
variety, is labeled according to whether it is a \emph{tancat} or not,
and, most importantly, is assumed to be hydrologically connected with a
single ditch.

The yearly management plan for these clusters, is described in
Appendix~\ref{sec-data-albufera-management}, and provides, for each day
of the year \(d\), the following:

\begin{itemize}
\tightlist
\item
  \(\text{Ir}_d\): boolean expressing whether the cluster is supposed to
  be irrigated (if true) on day of year \(d\).
\item
  \(\text{Dr}_d\): boolean expressing whether the cluster is supposed to
  be drained (if true) on day of year \(d\).
\item
  \(\mathcal H _d\): ideal water level (in \(\text{cm}\)) of the cluster
  on day of year \(d\).
\end{itemize}

As explained in Appendix~\ref{sec-data-albufera-management}, the
management plan depends on the rice variety and on the \emph{tancat}
label of a cluster.

In addition to these rather complex inputs, the model component requires
the following numerical parameters:

\begin{itemize}
\item
  \(k_\text{flow}\) (\emph{Ideal flow rate}). Rate at which water flows
  through rice paddies when these are being simultaneously irrigated and
  drained, with the overall level being kept constant. Expressed in
  \(\text{cm}\cdot\text{day}^{-1}\).
\item
  \(h_{\text{thres}}\) (\emph{Height threshold}). Maximum allowed water
  level for a cluster to be considered emptied, used in the calculation
  of draining/irrigation plan delays. Expressed in \(\text{cm}\).
\end{itemize}

Finally, it should be noted that the simulation involves random sampling
(see Subsection~\ref{sec-hbp-details}), so that the seed of random
number generation is an additional pseudo-parameter.

\subsection{Output}\label{output-2}

The output of this layer provides a simulation for the local hydrology
of each cluster, summarized in the table below

\begin{longtable}[]{@{}
  >{\raggedright\arraybackslash}p{(\columnwidth - 4\tabcolsep) * \real{0.3816}}
  >{\raggedright\arraybackslash}p{(\columnwidth - 4\tabcolsep) * \real{0.1842}}
  >{\raggedright\arraybackslash}p{(\columnwidth - 4\tabcolsep) * \real{0.4342}}@{}}
\caption{HBP model component
outputs}\label{tbl-hbp-output}\tabularnewline
\toprule\noalign{}
\begin{minipage}[b]{\linewidth}\raggedright
Variable
\end{minipage} & \begin{minipage}[b]{\linewidth}\raggedright
Units
\end{minipage} & \begin{minipage}[b]{\linewidth}\raggedright
Description
\end{minipage} \\
\midrule\noalign{}
\endfirsthead
\toprule\noalign{}
\begin{minipage}[b]{\linewidth}\raggedright
Variable
\end{minipage} & \begin{minipage}[b]{\linewidth}\raggedright
Units
\end{minipage} & \begin{minipage}[b]{\linewidth}\raggedright
Description
\end{minipage} \\
\midrule\noalign{}
\endhead
\bottomrule\noalign{}
\endlastfoot
\(h_{c,t}\) & \(\text{m}^3\) & Cluster's water height \\
\(I_{c,t}\) & \(\text{m}^3\) & Cluster's inflow \\
\(O_{c,t}\) & \(\text{m}^3\) & Cluster's outflow \\
\end{longtable}

\subsection{Detail}\label{sec-hbp-details}

\subsubsection{Ditch inflows}\label{sec-ditch-inflows}

The first step consists in breaking down the total inflow to the lake,
obtained through Equation~\ref{eq-hydro-balance-symbolic}, into inflows
from individual ditches. Each paddy cluster is assumed to communicate
with a single ditch (\emph{cf.} Section~\ref{sec-data-rice-clusters}),
and the flow through a ditch is assumed to be proportional to the total
area of clusters belonging to said ditch. Therefore, denoting by \(I_i\)
and \(A_i\) the inflow and area of ditch \(i\), and by \(I\) the total
inflow, we compute:

\begin{equation}\phantomsection\label{eq-ditch-inflow}{
I_i = \frac{A_i}{\sum _j A_j} \cdot I 
}\end{equation}

\subsubsection{Main Algorithm}\label{main-algorithm}

In the following, we focus on the set of clusters that communicate with
a specific ditch, whose inflow (estimated according to
Section~\ref{sec-ditch-inflows}) we denote by \(Q\). The essence of the
algorithm is to approximate as closely as possible the cluster's
\emph{ideal} inflows and outflows, with the constraint that the sum of
the actual outflows from all clusters must equal \(Q\).

Let us start by setting up some basic notation. We denote by:

\begin{equation}\phantomsection\label{eq-hbl-ideal-qts}{
h_c ^\text{id}(t),\quad I_c ^\text{id}(t),\quad O_c ^\text{id}(t),
}\end{equation}

the ideal water level, inflow and outflow of cluster \(c\) at time
(\emph{i.e.} day) \(t\), and by:

\begin{equation}\phantomsection\label{eq-hbl-real-qts}{
h_c ^\text{re}(t),\quad I_c ^\text{re}(t),\quad O_c ^\text{re}(t),
}\end{equation}

the corresponding real quantities. The area of cluster \(c\), that
provides the conversion between water volumes and column heights, is
denoted by \(A_c\). Precipitation and evapotranspiration water column
values are denoted by:

\begin{equation}\phantomsection\label{eq-hbl-petp}{
\text{P}(t),\quad \text{ETP}(t).
}\end{equation}

Finally, we will define below a \emph{plan delay} accumulated for
cluster \(c\) at time \(t\), denoted \(D_c(t)\). This is computed
recursively along the iterations of the algorithm, and its role will be
clarified below.

The local balance algorithm proceeds iteratively as follows. At day
\(t\), assume that \(h_c ^\text{re}(t-1)\) and \(D(t-1)\) have already
been computed; a single iteration consists of the following steps, which
we describe in full detail below:

\begin{itemize}
\tightlist
\item
  \textbf{\emph{Step 1.}} Recovering the \(h_c ^\text{id}(t)\),
  \(I_c ^\text{id}(t)\) and \(O_c ^\text{id}(t)\) from the irrigation
  and draining plan, applying the computed plan delay \(D_c(t-1)\).
\end{itemize}

\begin{enumerate}
\def\labelenumi{\arabic{enumi}.}
\setcounter{enumi}{1}
\item
  \textbf{\emph{Step 2.}} Computing \(h_c ^\text{re}(t)\),
  \(I_c ^\text{re}(t)\) and \(O_c ^\text{re}(t)\) enforcing the
  constraint \(\sum _c O_c ^\text{re}(t) = Q(t)\).
\item
  \textbf{\emph{Step 3.}} Computing the updated plan delays \(D_c(t)\).
\end{enumerate}

In order to initialize the iteration, we assume that at some initial
time, say \(t = 0\), we have \(D_c(t=0) = 0\)
\(h_c ^\text{re}(t=0) = h_c ^\text{id}(t=1)\).

\paragraph{Step 1: ideal balance}\label{step-1-ideal-balance}

The management data-set described
Section~\ref{sec-data-albufera-management} provides the ideal water
level for each cluster and for every day of the year. Denoting by
\(d(t)\) the day of the year corresponding to \(t\), the relevant entry
of the management data-set is that corresponding to day:

\begin{equation}\phantomsection\label{eq-hbl-yday-shifted}{
d^\prime_c(t) = d(t) - D_c(t-1),
}\end{equation}

where \(D_c(t-1)\) is the accumulated plan delay for thi cluster.
Therefore we set:

\begin{equation}\phantomsection\label{eq-hbl-ideal-height}{
h_c^\text{id}(t) = \text{Planned water level on day of year }d_c^\prime(t).
}\end{equation}

To compute ideal inflow and outflow, we require:

\begin{equation}\phantomsection\label{eq-hbl-ideal-balance}{
h_c^\text{id}(t) = \max(0,\,h_c^\text{re}(t-1)+\text{P}(t)-\text{ETP}(t)) + I_c ^\text{id}(t)-O_c ^\text{id}(t),
}\end{equation}

where we assume \(I_c ^\text{id}(t) >0\) and \(O_c ^\text{id}(t)>0\).
Clearly, Equation~\ref{eq-hbl-ideal-balance} alone does not individually
specify \(I_c ^\text{id}(t)\) and \(O_c ^\text{id}(t)\), but only their
difference \(\Delta _c (t) = I_c ^\text{id}(t) - O_c ^\text{id}(t)\). In
order to fix both these quantities, we first define\footnote{The
  condition is again evaluated using the the management data-set, where
  the relevant variables are the \texttt{irrigation} and
  \texttt{draining} columns. The cluster is understood to be in flux if
  both \texttt{irrigation} and \texttt{draining} are \texttt{TRUE}.}:

\begin{equation}\phantomsection\label{eq-hbl-ideal-flows-zero}{
\begin{split}
(I_c ^\text{id}(t))^{(0)}&= \begin{cases}
  k  & \text{cluster planned to be in flux on day of year } d_c^\prime(t) \\
  0 & \text{otherwise}
\end{cases},\\
(O_c ^\text{id}(t))^{(0)}&=(I_c ^\text{id}(t))^{(0)}-\Delta_c(t).
\end{split}
}\end{equation}

and, in order to ensure that flows are positive\footnote{The positivity
  of \(I_c ^\text{id}(t)\) below is ensured by the fact that
  \((O_c ^\text{id}(t))^{(0)}\geq-\Delta_c(t)\) by construction.}, we
finally set:

\begin{equation}\phantomsection\label{eq-hbl-ideal-flows}{
\begin{split}
O_c ^\text{id}(t) &= \max\lbrace(O_c ^\text{id}(t))^{(0)},\,0\rbrace\\
I_c ^\text{id}(t) &= O_c ^\text{id}(t) + \Delta_c ^\text{id}(t)
\end{split}
}\end{equation}

\paragraph{Step 2: real balance}\label{step-2-real-balance}

Real flows are obtained from ideal ones
(Equation~\ref{eq-hbl-ideal-flows}) in such a way to satisfy the
constraint:

\begin{equation}\phantomsection\label{eq-hbl-outflow-sum-constraint}{
\sum _c O_c ^\text{re}(t) = Q(t),
}\end{equation}

where the right-hand side is the total ditch flow computed earlier
(Section~\ref{sec-ditch-inflows}). At each time-step \(t\), the
cluster's index set is randomly permuted\footnote{With some abuse of
  notation, we assume the indexes \(c\) and \(c'\) in
  Equation~\ref{eq-hbl-real-flows} to be sorted according to this random
  permutation.}, and the real flows are calculated as:

\begin{equation}\phantomsection\label{eq-hbl-real-flows}{
\begin{split}
O_c ^\text{re}(t) &= \min \lbrace O_c ^\text{re}(t),\,Q(t)-\sum_{c'<c}O_{c'} ^\text{re}(t) \rbrace +
\dfrac{\max \lbrace 0, Q(t)-\sum_{c'}O_{c'} ^\text{re}(t)\rbrace}{\text{number of clusters}},\\
I_c ^\text{re}(t) &= \max\lbrace I_c ^\text{id}(t)-O_c ^\text{id}(t) + O_c ^\text{re}(t),\,0 \rbrace
\end{split}
}\end{equation}

In words, clusters are emptied in a random order within the allowed
capacity of the corresponding ditch (\emph{i.e.} its actual total flow)
- if the sum of ideal outflows is less than capacity, the remaining
outflow is equally shared among clusters. Using
Equation~\ref{eq-hbl-real-flows}, we finally determine the real water
level achieved as:

\begin{equation}\phantomsection\label{eq-hbl-real-balance}{
h_c^\text{re}(t) = \max(0,\,h_c^\text{re}(t-1)+\text{P}(t)-\text{ETP}(t)) + I_c ^\text{re}(t)-O_c ^\text{re}(t),
}\end{equation}

to be compared with Equation~\ref{eq-hbl-ideal-balance}.

\paragraph{Step 3: updating the plan
delay}\label{step-3-updating-the-plan-delay}

The purpose of the plan delay \(D_c(t)\) is to allow all clusters to be
emptied as required by the ideal management plan, which may be hindered
on the originally scheduled days by the first of
Equation~\ref{eq-hbl-real-flows}, since this sets to zero the real
outflows for some clusters whenever the ditch's flow is saturated.

The updated value \(D_c(t)\) is obtained as follows. If \(d(t)\) (the
\emph{actual} day of year) is outside of the window
\(W = [\text{20th of April},\,\text{15th of October}]\), we reset all
\(D_c(t) = 0\). Otherwise, if \(h_c ^{\text{id}}(t)>0\) or
\(h_c ^{\text{re}}(t)<H_{\text{thres}}\), the plan delay is unchanged
for this cluster: \(D_c(t) = D_c(t-1)\). Finally, if
\(h_c ^{\text{id}}(t)=0\) but \(h_c ^{\text{re}}(t)>H_{\text{thres}}\),
we set \(D_c(t) = D_c(t-1) + 1\).

\section{CA: Chemical Applications}\label{ca-chemical-applications}

TBD.

\section{CT: Chemical Transport}\label{ct-chemical-transport}

This model component describes the evolution of chemical masses in the
three compartments of foliage, water and sediment. This is obtained by
solving the system of differential equations that describes the dynamics
of chemicals, through a semi-analytic approach with suitable
approximations and simplifications.

\subsection{Input}\label{input-3}

TBD.

\subsection{Output}\label{output-3}

TBD.

\subsection{Details}\label{details-1}

The evolution of masses in the foliage, water and sediment compartments
is described by the following system of differential equations:

\$\$

\begin{split}

\dfrac{\text d{m}_f}{\text d t} & = -(k_f+w)m_f+a_f\\
\dfrac{\text d{m}_w}{\text d t} & = -(k_w+d_w+s+\frac{O}{V})m_w+d_sm_s+wm_f+a_w-\sigma(\frac{m_w}{\rho V}) \\
\dfrac{\text d{m}_s}{\text d t} & = (d_w+s)m_w-(k_s+d_s)m_s+a_s+\sigma(\frac{m_w}{\rho V})\\

\end{split}

\$\$ \{\#eq-ct-ode\}

where:

\begin{itemize}
\tightlist
\item
  \(a_{f,w,s}\) are the mass application rates of the chemical in the
  three compartments,
\item
  \(k_{f,w,s}\) are the degradation rates of the chemical in the three
  compartments,
\item
  \(d_w\) and \(d_s\) are the water-sediment diffusion rates,
\item
  \(s\) is the settling rate,
\item
  \(w\) is the washout rate,
\item
  \(O\) is the outflow rate of the rice field,
\item
  \(V\) is the volume of water in the rice field,
\item
  \(\rho\) is the chemical solubility in water,
\item
  \(\sigma(x)\) is a function (not further specified, see below) that
  grows quickly for \(x > 1\), and vanishes for \(x \leq 1\). This
  function accounts for solubility.
\end{itemize}

Strictly speaking, all these terms have instantaneous time
dependence\footnote{ This observation also includes pure
  physico-chemical ``constants'' such as degradation rates, where the
  time dependence would stem from \emph{temperature} dependence.}. Apart
from making the \textbf{?@eq-ct-ode} hard to attack by analytic means,
such a dependence is troubling because we don't have access to the exact
time dependence of the majority of these terms (\emph{e.g.} outflow,
volume or chemical applications), but we only know daily
average/cumulative values.

On the other hand, if all the terms were constant, and if we could
neglect \(\sigma\), the solution of \textbf{?@eq-ct-ode} would be
immediate, as the corresponding system becomes a linear ODE with
constant coefficients, in addtion whose eigenvalues and eigenvectors can
be computed explicitly. What we use in practice is an intermediate
semi-analytic approach, which we describe in the following.

\part{User Manual}

\chapter{The ERAHUMED DSS User
Interface}\label{the-erahumed-dss-user-interface}

\chapter{\texorpdfstring{The \texttt{\{erahumed\}} R
package}{The \{erahumed\} R package}}\label{the-erahumed-r-package}

\bookmarksetup{startatroot}

\chapter*{References}\label{references}
\addcontentsline{toc}{chapter}{References}

\markboth{References}{References}

\phantomsection\label{refs}
\begin{CSLReferences}{1}{0}
\bibitem[\citeproctext]{ref-martinez2024pharmaceutical}
Martı́nez-Megı́as, Claudia, Alba Arenas-Sánchez, Diana Manjarrés-López,
Sandra Pérez, Yolanda Soriano, Yolanda Picó, and Andreu Rico. 2024.
{``Pharmaceutical and Pesticide Mixtures in a Mediterranean Coastal
Wetland: Comparison of Sampling Methods, Ecological Risks, and Removal
by a Constructed Wetland.''} \emph{Environmental Science and Pollution
Research} 31 (10): 14593--609.

\end{CSLReferences}

\cleardoublepage
\phantomsection
\addcontentsline{toc}{part}{Appendices}
\appendix

\chapter{Input Data}\label{input-data}

\section{Hydrological data}\label{sec-data-hydro}

\section{Meteorological data}\label{sec-data-meteo}

\section{Albufera Rice Paddies
Management}\label{sec-data-albufera-management}

\section{Storage curve and P-ETP function}\label{sec-storage-curve}

\section{Definition of rice clusters}\label{sec-data-rice-clusters}




\end{document}
